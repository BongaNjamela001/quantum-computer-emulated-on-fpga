\chapter*{Abstract}

In the field of quantum information processing, quantum computers exploit the principles of quantum mechanics to manipulate quantum bits, or qubits, to solve complex problems which require techniques that are beyond the capabilities of classical computers. However, other properties of qubits such as decoherence make it difficult and exorbitantly expensive to employ the full capabilities of quantum computers. Accordingly, the fabrication of a quantum computer must satisfy certain criterion in order for it to successfully process quantum information. To circumvent the challenges that arise from fabricating quantum computing hardware, intrinsic parallelism of high performance classical computing software can be used to simulate some of the quantum algorithms performed by the quantum computer. To allow researchers to execute quantum circuits, the following paper proposes the design of an FPGA-based quantum computer emulator, including a model of a quantum interface that facilitates quantum and classical communication between a classical processor and a quantum computer. 

Qubits were modelled to satisfy DiVincenzo's five criteria which requires qubits to be initialised to fiduciary and well-characterised quantum states for sufficiently long decoherence times. The design also aims to satisfy the quantum communication criteria for establishing qubit transfers between a source and a receiver. For quantum communications to exists, the interface must have the ability to convert flying-qubits to stationary qubits and map physical qubits to logical qubits in order to perform quantum algorithms. Emulations of quantum communication protocols are of particular interest due to the no cloning theorem which prevents qubits from being duplicated. 

To overcome the no cloning clause, the proposed design implemented the quantum teleportation algorithm for transferring quantum information through entangled qubits. This was conducted using the proposed model of a quantum interface whereby coherent laser pulses for transmitting single-photon qubits through a quantum key distribution network were modelled using light intensity from eight LEDs. Correspondingly, InGaAs  based avalanche photodoide detectors for converting flying-qubits to stationary qubits were modelled using photoresistors, or LDRs, to represent qubit states in the computational basis using resistances. The paper also proposes two transmission methods. In the first method, flying-qubits were transferred directly in the computational basis states such that a high resistance state was related to the $\ket{1}$ basis state, a low resistance state was related to the $\ket{0}$ state. In the second method, LEDs were used to model the transfer of the Hilbert space of qubits. Qubit transmission through laser pulse control was modelled using a STM32 microcontroller.

On the edge of the QKD network, a Nexys A7 FPGA was used to emulate the execution of quantum circuits. The number of resources required was increase with the number of qubits and the number of quantum gates required to perform a quantum algorithm. Specifically, the FPGA used FIFO macros to store detected qubits and map them to logic qubits, then quantum gate operations were modelled using LUTs and DSP slices to perform the associated unitary matrix operations. A universal set of quantum gates was also defined in line with DiVincenzo's criteria. The set was determined according to the quantum algorithms that were executed by the emulator. In addition to the quantum teleportation algorithm, the proposed design allows users to perform the 3-qubit quantum fourier transform, the quantum factoring algorithm and the quantum search algorithm. A study of the quantum circuits showed that the necessary gates for successfully emulating these algorithm were the Hadamard, Pauli single-input gates, as well as the controlled multi-qubit gates. Emulations of the Hadamard and controlled gate required the most FPGA resources and power due to principles of superposition and entanglement.
