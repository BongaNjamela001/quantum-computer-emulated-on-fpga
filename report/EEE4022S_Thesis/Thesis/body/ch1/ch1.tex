\chapter{\label{ch:intro} Introduction}

Moore's Law accurately predicted the exponential increase in transistors on integrated circuit chips, however, James R. Powell indicated that a persuasive argument from quantum mechanics based on the Heisenberg principle, defines an eventual limit to the miniaturization of transistors that can be achieved \cite{powell2008}. Conversely, quantum mechanics provides advantages that can be exploited to improve the computational capabilities by employing principles of superposition, entanglement and interference to represent different states of information. However, physical control and management of quantum systems requires vast resources that make it difficult to realise quantum computers and exploit the performance in finding solutions to complex problems. In recent times, rapid development and scaling of advanced classical computers that use transistors to represent bits as a 0 or 1 and perform functions using elementary logic elements, has made the simulation of Feynman's quantum computers marginally achievable. Particularly, Field Programmable Gate Arrays (FPGAs), based on complementary metal-oxide semiconductor (CMOS) technology, offer high performance, low power consumption, reduced cost, flexibility and scalability which is suitable for performing the sparse matrix operations that are necessary for exploiting the properties of quantum systems. This paper focuses on the emulation a quantum computer on a FPGA in order to perform quantum algorithms without having to build an actual quantum computer. 

Using the quantum state of particles to represent information forms the foundation of quantum computing - a computing paradigm that uses the principles of quantum physics to represent and compute information as \textit{qubits}. From the principle of superposition, qubits can exist in all eigenstates simultaneously, unlike classical bits which can only be in a state of 0 or 1, and not both at the same time. Additionally, quantum entanglement allows qubits to be manipulated simultaneously, whereas classical bits can only be manipulated individually per operation. Exploiting these principles gives quantum computing a computational advantage where multiple instances of data can be manipulated simultaneously, thereby reducing the size of the instruction set as part of the architecture of a quantum computer. Another advantage of quantum computers is that, since qubits can exist in multiple states at the same time, the number of qubits required to perform a computation is exponentially smaller than the number of classical bits that would be required to perform the same operation. These benefits account for the growing interest in quantum computing as means to overcome the limits of transistor growth as predicted by Moore's Law. Consequently, the use of quantum computer can reduce the time complexity of many processes that require exceptional performance to execute tasks. 

We propose an implementation of the Quantum Fourier Transform (QFT), which is a critical subroutine in the quantum algorithms that are executed on the quantum computer. The paper follows an FPGA system design flow to implement Shor's factoring algorithm and Grover's search algorithm in the computational basis of a quantum system. Quantum algorithm subroutines are represented as quantum circuits, which use universal \gls{quantum gate}s to perform the \gls{unitary operation}s that transform quantum states in the computational basis. FPGAs are suitable for emulating quantum gate operations due to the large number of combination logic blocks and flip-flops for mapping qubits and their transformations to operations on classical bits. 

In the physical realisations of quantum computers, single-photon qubit emissions are typically induced by the energy transitions of electron and hole pairs of a \gls{quantum dot} \cite{hours2003single, stievater2001rabi}. In addition to emulating a quantum computer, the paper describes the methodology for modelling the instantiation of single-photon qubits and the transmission as flying qubits in a quantum communication channel. The proposed system uses a microcontroller to model the initialisation of ideal qubits according to DiVincenzo's criteria for the realisation of quantum computers \cite{divincenzo2000physical}. The microcontroller uses a STM32F051C6 target to run a C program that simulates the behaviour of a laser pulse modular by toggling light emitting diodes (LEDs) with a wavelength of about $\SI{700}{\nano\meter}$ and a frequency that is related to the decoherence time of electrons. In  the modelled physical quantum systems that use interferometry and spectroscopy, incident photons from a monochromatic laser pump induce energy level transitions of electrons and holes in the conduction band of a GaAs quantum dot as described by Li, Zhang and Wu and Eisaman \cite{li2022control, eisaman2011invited}. In this design, photoresistor were used to model the operation of avalanche photodiode detectors (APDs) by correlating the computational basis states to resistance states separated by the cut-off and saturation regions of an NPN-type Bipolar Junction Transistor (BJT). 

The operation of the quantum interface for transferring qubits and classical information was tested using the quantum teleportation algorithm where Alice sends one half of an entangled qubit pair to Bob before performing Bell state measurements on message qubit. The quantum teleportation protocol is used in quantum key distribution (QKD) network to securely transfer keys between a source and receiver. Shor's algorithm was performed to find the order of the integer 21. This algorithm was expected to require the most resources for emulation on the FPGA due to the large number of quantum gate operations involved in the execution of the quantum circuit. Lastly, the design explored implementation of the quantum search algorithm on a database with 4 entries.


\section{\label{sec:background}Background}

In 1982, Richard Feynman identified the complexity of simulating the probabilistic nature of quantum physical phenomena using classical computers \cite{feynman2018simulating}. Feynman queried the viability of physics simulations on quantum computing based on the \textit{hidden-variable problem} which states that it is impossible to represent the results of quantum mechanics with a Turing machine. From this query, Feynman established the foundation for quantum computers by proposing the discretisation of space as an approximation of the phenomena of field theory. In other words, universal quantum computers can describe every finite quantum mechanical system \textit{exactly} by supposing that at each point in space-time, the system has only two possible states known as quantum bits or \textit{qubits} \cite{feynman1981simulating}. Qubits are the fundamental unit of quantum information which exploit quantum mechanical principles of superposition, entanglement, decoherence and the no-cloning theorem of quantum states to perform complex computations in exponentially quicker times than classical bits.

In recent times, the field of quantum information has seen rapid growth with companies such as D-Wave, Google Quantum AI, IBM and Intel having developed multi-qubit quantum computing platforms that can be used by the public to build quantum applications for diverse problems such as drug discovery in the pharmaceutical industry and portfolio optimisation in business computing \cite{DWave00}. D-Wave quantum systems use a process called \textit{\gls{quantum annealing}} that searches for solutions to a problem using the amplitudes of the quantum wave functions described in quantum mechanics \cite{DWave01}. In March 2017, IBM released \textit{Qiskit}, which is a cross-platform development toolkit for building and transpiling quantum circuits on Noisy Intermediate-Scale Quantum (NISQ) processors that use superconducting qubits \cite{IBM00}. In 2023, Google Quantum AI achieved the first-demonstration of a logical qubit NISQ prototype that promises reduction in errors by increasing the number of ancilla qubits in a scheme known as \textit{quantum error correction} \cite{google2023suppressing}. Quantum error correction arises from the difficulty of fabricating qubits with sufficiently long decoherence times. For a quantum computer to be scalable, it needs a large number of qubits to implement quantum error correction and perform useful computations. The trajectory and rate of growth of the field of quantum computing depends on finding reliable methods for correcting errors that make it difficult to shield and coherently control the dynamics of quantum mechanical properties such as electron spin and photon polarisation \cite{gill2024quantum}. Ultimately, interaction of qubits with their immediate environment poses challenges for creating efficient hardware that can suitably facilitate quantum computations without degrading qubits \cite{tanaka2024single}. This requires quantum computers to be rigorously isolated in order to retain quantum states and perform useful computations. 

The largest quantum computers by IBM and Atom Computing, use superconducting wires that are cooled to extremely low temperatures to maintain the quantum states of qubits \cite{wilkins2024record}. In 2023, IBM released the first quantum computing chip with 1121 superconducting qubits arranged in a honeycomb pattern. In the same year, Atom Computing demonstrated the ability to measure the quantum state of specific qubits and detecting errors that occuring during computation on quantum computer with 1180 qubits \cite{swayne2023atom}. Challenges faced by these institutions in fabricating qubits lead to costly and limited access to quantum computing platforms \cite{belfore2024scalable}. Physicists and engineers have made efforts to annex the propensity of qubits to degrade by coaxing the \textit{physical qubits} encoded in a superconducting circuits to work cohesively to represent one qubit of information, or \textit{logical qubit} \cite{castelvecchi2023ibm}. Alternatively, quantum computing emulations and simulations provide a means to enhance research and industry efforts that require \textit{quantum supremacy} for solving \gls{NP-complete} problems that classical systems cannot resolve. Additionally, emulated and simulated quantum computers may present experts with platforms for performing operations on qubits that are currently not achievable on existing quantum computers \cite{belfore2024scalable}. Operations on qubits are represented by \textit{quantum gates} - analogous to classical logic gates implemented on FPGAs and other classical computing devices. 

Modelling quantum computers through emulations is not without its own challenges. For example, storing a the quantum state of a qubit requires two floating point numbers with imaginary and real parts to represent the amplitudes of the wave functions. With each floating point typically presented by 32 to 64 classical bits, modelling a quantum computer with 1000 qubits would require $2^{1000}$ bits or more than $10^{300}$ bytes in total. Comparatively, the current global data storage capacity is only $10^{21}$ bytes in total \cite{taylor2023storage}. As such, the problem of modelling quantum computers can be classified as a \textit{dense linear algebra} of the \textit{Seven Dwarfs of high performance computing} presented by researchers at the University of California at Berkeley as a method for capturing common computation and communication patterns for a wider range of applications \cite{asanovic2006landscape}. 

Quantum computers can be modelled for systems with a small number of qubits. For example, using 32 bits to represent the floating-point amplitudes of a quantum state, a quantum computer with 27 qubits can be comfortably modelled with 1 GB of memory \cite{brown2010using}. This implies that the demand for intensive computational resources increases as the number of qubits of the quantum computer that is being modelled also increase. However, quantum emulations and simulations are very useful for verifying quantum algorithms and their associated quantum circuit representation. In this paper, a design and implementation of a quantum computer emulated on a FPGA is proposed. Although an FPGA cannot directly modelled quantum mechanical properties based on the limitations identified by Feynman, the reconfigurability, multiple combinational logic blocks and intrinsic parallelism of the device make it highly suitable for emulating the quantum circuits. In particular, this paper focuses on the emulation of the quantum teleportation algorithm in QKD applications, the 3-qubit quantum fourier transform, and the quantum factoring algorithm, and the quantum search algorithm.

%@@@@@@@@@@@@@@@@@@@@@@@@@@@@@@@@@@@@@@@@@@@@@@@@@@@@@@@@@@@@@@@@@@@@@@@@@@@@@@@@@@@@@@@@@@@@@@@@@@@@@@
\section{\label{sec:probdesc}Problem Description}

Quantum computing offers significant advantages over classical computing, leveraging principles such as superposition and entanglement to solve complex problems at exponentially faster rates. However, the construction of physical quantum computers is expensive, complicated by issues like decoherence and the need for extremely low temperatures. To address these challenges, emulating quantum computers on classical hardware, such as FPGAs, offers a practical alternative. This project focuses on designing an FPGA-based quantum computer emulator, aiming to simulate key quantum algorithms such as quantum teleportation, the 3-qubit quantum fourier transform (QFT), and Grover's search algorithm. The emulator also models quantum communication through interfaces that facilitate qubit transmission between classical and quantum systems. By doing so, the proposed system provides a viable platform for testing and experimenting with quantum algorithms without the need for costly physical quantum computing hardware.

%@@@@@@@@@@@@@@@@@@@@@@@@@@@@@@@@@@@@@@@@@@@@@@@@@@@@@@@@@@@@@@@@@@@@@@@@@@@@@@@@@@@@@@@@@@@@@@@@@@@@@@
\section{\label{sec:focus}Focus and Objectives}

The following paper focuses on the design and implementation of an FPGA-based quantum computer emulator. The paper also focuses on modelling the quantum interface between a classical computer and a quantum computer. The aim of the design is to model key quantum algorithms, including quantum teleportation, the 3-qubit Quantum Fourier Transform (QFT), and the quantum search algorithm, and the quantum factoring algorithm using a combination of classical and quantum information systems. By modelling the quantum communication interface, the paper aims to demonstrate efficient quantum state transmission and interactions between classical and quantum computers.
%@@@@@@@@@@@@@@@@@@@@@@@@@@@@@@@@@@@@@@@@@@@@@@@@@@@@@@@@@@@@@@@@@@@@@@@@@@@@@@@@@@@@@@@@@@@@@@@@@@@@@@

\section{\label{sec:methology_overview}Methodology Overview}
%Nice to have a brief methodology overview in CH1.

%\begin{itemize}
%\item Some items
%\item In a bulleted list
%\end{itemize}

\section{Scope and Limitations}

The scope of this research is limited to the emulation of quantum circuits on FPGA platform and does not attempt to physically implement quantum computing hardware. The paper also does not attempt to simulate quantum mechanical properties of quantum states directly. The emulator provides a platform for researchers explored quantum-classical communication protocols through sotware and hardware interfaces, using LEDs and photoresistors to model qubit transmission. The project also exploits the intrinsic parallelism in FPGAs provided by resources such as DSP slices, FIFOs, and LUTs to emulate quantum gate operations and transmission protocols. Some of the limitations of this paper include:

\begin{itemize}
	\item 
	The emulation is limited by the resources of the FPGA board. In particular, this paper is limited by the capabilities of the Xilinx 7-series FPGA.
	\item 
	Computational accuracy is limited by the hardware which can lead to errors in state fidelity.
	\item 
	The scope is limited to a emulating up to 5 qubits in a quantum register.
\end{itemize}

\section{Plan of Development}

\textbf{Chapter 1} provides an overview of quantum computing and presents the motivation for the study, the focus, scope and limitations.

\textbf{Chapter 2} introduces the mathematical and quantum mechanical principles that govern the behaviour of quantum information systems.  This includes the foundational postulates of quantum mechanics, quantum states, unitary operations, an the role of qubits in quantum circuits.

\textbf{Chapter 3} outlines relevant work in quantum computing, quantum communication systems, and FPGA-based emulation, highlighting the challenges of real-world quantum computing implementations.

\textbf{Chapters 4} describes the methodology for designing and implementing the quantum computer emulator on an FPGA, including system requirements, modular subsystems, and the development process for quantum algorithms.

\textbf{Chapters 5} provides the technical design details of each subsystem, including quantum state initialization, quantum gate emulation, and the communication protocols between the classical processor and quantum systems.

\textbf{Chapter 6} presents the results of the emulated algorithms and compares them with theoretical expectations, including performance benchmarks of the quantum emulator on FPGA.

\textbf{Chapter 7} concludes the findings of the research and provides recommendations for future improvements and applications of FPGA-based quantum emulators.
