\chapter{\label{app:myappendix}GA Assessment}

\section{\label{app:sec:myappendix}GA1: Problem Solving}

The physical realisation of quantum computers poses challenges that make it difficult to harness the performance enhancing capability that they offer.  These challenges make it very expensive to implement actual quantum computers. The aim of the project is to emulate a quantum computer on a FPGA. I began by formulating a theoretical framework for understanding of quantum computing and its core principles. Conceptually, quantum computing is based on mathematics and physics, numerical analysis, data analysis, statistics and formal aspects of computer and information science. This is because quantum computations require a deep understanding of advanced linear algebra, quantum mechanics,  quantum algorithms such as Shor’s algorithm and Grober’s Search algorithm, as well as quantum information which draws some of its concepts from classical computing using bits.

Emulating a quantum computer requires engineering specialist knowledge that provides theoretical frameworks and bodies of knowledge.

\section{GA4: Investigations, Experiments, and Data Analysis}

I demonstrated competence to conduct investigations of the complex engineering problem of implementing a quantum computer through quantum hardware, quantum simulations and emulations. Through extensive research methods, I was able to write the literature review exploring previous emulations and simulations of quantum computers. From the knowledge acquired in the investigation, I was able to design an experiment which includes a quantum communication channel that uses a microcontroller to transfer “flying qubits” through a quantum channel to the FPGA-emulated quantum computer. In the quantum communication experiment, LEDs were used to model GaAs-based quantum dot single-photon emission and LDRs were used to model the specialised detectors at the receiver’s end of the channel. These experiments allowed me to investigate the fidelity of the quantum channel and the methods used for initialising and transmitting flying qubits. Before investigating how quantum algorithms such as Shor’s algorithm and Grober’s search algorithms could be implemented on the FPGA, I investigated the criteria for the implementation of a quantum computer as proposed by DiVincenzo. This knowledge allowed me to ensure that mappings of classical bits to qubits satisfied what is expected from a real quantum computer. Experiments were designed to test the performance gained from implementing quantum algorithms compared to  equivalent classical functions. Shor’s algorithm was expected to show polynomial improvements, while Grober’s search algorithm was expected to be only slightly faster than standard database search algorithms. 

\section{GA5: Use of Engineering Tools}

I have used:
- STM32 Microcontroller for initialising and transmitting qubits to the quantum module. Microcontrollers are an essential tool in embedded systems engineering
- Atollic Software for developing the Embedded C program for initialising quantum data on the microcontroller to model laser control for single-photon qubit transmissions.
- The Nexys A7 FPGA was a key part of the design and the main primary hardware tool used
- Xilinx Vivado for synthesis and simulation with SystemVerilog
- MATLAB for simulations and generation of pseudo random numbers as quantum state probabilites
- Electronics components such as LDRs, on-board IO pins on the FPGA  

\section{GA6: Professional and Technical Communication}

I communicated with my supervisor through MS Teams, email and in-person meetings where we were able to discuss the concepts that were required to emulate the quantum computer on FPGA. I am in the process of collecting data from the system and tabulating the results in the report. I am using Latex to write the report and diagrams for illustrations are creating using CAD tools. I used a Gantt chart to manage and track the progress of the project. Additionally, I used GitHub as a version control tool for managing code and storing data.

\section{GA 8: Individual Working}

After our discussions with the supervisor, I was able to demonstrate competence to work effectively as an individual. I have annotated the ECSA Code of Conduct to demonstrate knowledge of professional ethics.

\section{GA 9: Independent Learning Ability}

I demonstrated competence to engage in independent learning through well developed learning skills.   I had to learn about quantum computing concepts independently. At times, it felt that the challenge was insurmountable but focusing intensely on the project and reading through multiple resources allowed me to gain the confidence required to proceed.

To access documents supporting the GA assessment, please visit the GitHub repo at: \url{https://github.com/BongaNjamela001/quantum-computer-emulated-on-fpga}.