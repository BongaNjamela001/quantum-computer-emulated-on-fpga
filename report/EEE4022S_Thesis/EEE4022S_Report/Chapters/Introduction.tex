% ----------------------------------------------------
% Introduction
% ----------------------------------------------------
\documentclass[class=report,11pt,crop=false]{standalone}
\input{../Style/ChapterStyle.tex}
\input{../FrontMatter/Glossary.tex}
\begin{document}
% ----------------------------------------------------
\chapter{Introduction}
%\epigraph{Philosophers have hitherto only interpreted the world in various ways; the point is to change it.}%
%    {\emph{---Karl Marx}}
%\vspace{0.5cm}
% ----------------------------------------------------

%\lipsum[1]

Computational capabilities of classical computers that use binary to describe information are limited by physical constraints of power consumption and the amount of physical space that each element of information occupies. Moore's Law accurately predicted the exponential increase in transistors on integrated circuit chips, however, James R. Powell indicated that a persuasive argument from quantum mechanics based on the Heisenberg principle, defines an eventual limit to the miniaturization of transistors that can be achieved \cite{powell2008}. On the contrary, quantum mechanics also provides advantages that can be exploited to improve the computational capabilities by employing principles of superposition and quantum entanglement to represent different states of information. 

Using the quantum state of particles to represent information forms the basis of quantum computing - a computing paradigm that uses the principles of quantum physics to represent and compute information as \textit{qubits}. From the principle of superposition, qubits can be in a state of 0 and 1 simultaneously, unlike classical bits which can only be in a state of 0 or 1, and not both at the same time. Additionally, quantum entanglement allows qubits to be manipulated simultaneously, whereas classical bits can only be manipulated individually per operation. Exploiting these principles gives quantum computing a computational advantage where multiple instances of data can be manipulated simultaneously, thereby reducing the size of the instruction set as part of the architecture of a quantum computer. Another advantage of quantum computers is that, since qubits can exist in multiple states at the same time, the number of qubits required to perform a computation is exponentially smaller than the number of bits that would be required to perform the same operation using classical bits. These benefits account for the growing interest in quantum computing as means to overcome the limits of transistor growth as predicted by Moore's Law. Consequently, the use of quantum computer can reduce the time complexity of many processes that require exceptional performance to execute tasks. The following paper describes the design and emulation of a quantum computer on a field programmable logic gate array (FPGA) to simulate quantum acceleration of a classical computer. 

Quantum acceleration




\section{Background}
\lipsum[1]

\section{Objectives}
\lipsum[1]

\section{System Requirements}
\lipsum[1]

\section{Scope \& Limitations}
\lipsum[1]

\section{Report Outline}
\lipsum[1]

% ----------------------------------------------------
\ifstandalone
\bibliography{../Bibliography/References.bib}
\printnoidxglossary[type=\acronymtype,nonumberlist]
\fi
\end{document}
% ----------------------------------------------------